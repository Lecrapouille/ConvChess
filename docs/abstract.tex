\cleardoublepage

\begin{center}
	\huge{\textbf{Abstract}}
\end{center}

How well can one play chess without domain knowledge?  
Most such systems play so poorly that the question has not
been explored much.  Chess is a deterministic game, and it is assumed
that it would be solved by extensive search-based paradigms.  Though such 
approaches today perform far above human levels, such methods are weak in 
generalizing strategies, since the ideas cannot be extended to ``similar 
positions''. On the other hand, strong human players have strategies that seem 
to be based on pattern-driven implicit knowledge rather than explicit search.  
Can such approaches contribute to machine chess?

In this thesis we explore this question by training a chess machine that
learns merely by observing several thousand unannotated games between good
human players, along with outcome. The system uses the moves played in each
position as an oracle and uses multiple Convolutional neural networks to learn 
representations for the board.  These models are then tuned using 
back-propagation for a) learning the piece to be moved, b) learning which move 
to make given a piece, and c) learning to estimate the win-probability for a 
given board. We had very limited expectations of the system, and worried that 
it may not even learn the rules of the game. However, we found that the system 
not only plays legally correct moves (including complex situations such as 
castling, en passant or promotion), but finds fairly good moves in almost all 
board situations (the move actually played in the test set is almost always one 
of the top ten moves). Further, the evaluation function, which is trained by
discounting outcomes based on distance from the end, correlates with the 
conventional material heuristic evaluation function.  The system is able to
beat a fairly decent machine (Sunfish, \cite{sunfish}) in 21.6\% of games, even 
without any opening book.
