\chapter{Introduction}
\label{chap:introduction}
We associate being able to play a good game of chess with someone's mental 
ability. For this reason, when the study of `Artificial Intelligence' came into 
existence, chess was one of the earliest games to be studied. Alexander Kronrod, 
a Russian AI researcher, said ``Chess is the Drosophila of AI'' 
\cite{mccarthy1990chess}. He made an analogy between experiments on fruit fly to 
study inheritance to that of computers playing the game of chess to display 
intelligence. \citet{mccarthy} says that Artificial Intelligence has two tools 
of solving problems. One is to use methods inspired by human introspection, and 
other is to invent it completely using ideas of computer science independent of 
how humans do it. Chess programming lies at the intersection of the two. 
Although much of the mental computation done by the chess players is invisible 
to the outside observer and even to the player, while the parts of the game 
tree being explored are consciously accessible. This nature was what inspired 
much of the research community in the early stages of AI to take up computer 
chess as a scientific study.\\

It was in the 1970s that brute-force programs started to dominate 
that required higher set of computational resources rather than human 
intelligence like mechanisms, adding a competitive and commercial aspect to the 
domain. This eventually led to a loss of interest in computer chess of the AI 
community who valued human-like intelligence for machines as a scientific 
domain. 

This brute force way of computers playing chess, in the course of 
time, led Deep Blue, which was an IBM computer, to beat Garry Kasparov who was 
then the reigning world champion. It also led to a disagreement between a 
number of AI pioneers and researchers whether this was an important feature of 
human intelligence. Meanwhile the study of machine learning became popular 
amongst the community focusing on other interesting problems related to three 
major learning paradigms-- supervised learning, unsupervised learning and 
reinforcement learning. However, machine learning in chess has seldom been used 
except addressing several disjoint issues, for instance learning evaluation 
weights of various handcrafted features like piece 
values\cite{beal1997learning}, piece-square values\cite{beal1999learning} and 
mobility, while others have been used to learn opening book moves specifically 
\cite{hyatt1999book}. Some other programs which motivate our work are mentioned 
in chapter~\ref{chap:background}. 

\section{Motivation}
In this section, we look at what motivates us to solve the problem of chess as a 
pattern recognition task as opposed to the conventional approaches of 
accomplishing it as a predominantly search problem. We discuss various 
studies which suggest that most of the chess computer AIs that compete with the 
best human chess players don't probably play it the actual way humans do. 
Specifically, we define the task of playing chess in a more human way using a 
pattern recognition system in section~\ref{section:chess-as-pr}.

\subsection{How humans play chess?}
Adriaan de Groot, a Dutch chess master and a psychologist, after a deep 
statistical and interpretative analysis of chess players' transcripts of verbal 
utterances, their eye gaze movement and interviewing a number of beginner and 
master level chess players concluded that all players usually examine 40-50 
positions before playing a move. The difference however 
is that the master level players develop pattern recognition skills from 
experience which helps them examine only a few lines of play in much greater 
depth while ignoring the poor moves, where the beginners spend a lot of their 
time \cite{de1996perception}. Another evidence of why this is how humans 
play chess is that humans, especially at the master level, are capable of 
recognizing familiar arrangements of board or specific sets of pieces 
from experiences, rather than random arrangements of the same pieces  
\cite{chase1973perception}. Also unique to humans is the capability to learn 
from experience. This learning can both be ability to recognize patterns from 
the past games or it could be the ability to learn the weaknesses of the 
opponent or learn from his/her own strategic mistakes made.

\subsection{Learning to play chess}
\citet{ericsson2007making} says that scientific research that 
looked at extraordinary and exceptional performance in a number of fields shows 
that ``experts are always made, never born''. The author studied expertise in 
wide variety of domains ranging from chess to cooking, before concluding that 
the amount and quality of practice were the key factors in the level of 
performance achieved. The author also claims that a minimum of ten years of 
intense training is required before even the most gifted talents can 
win international competitions \cite{ericsson2006cambridge}. An aspect of this 
intense training and practice is that it is deliberate, meaning it 
consists of considerable and sustained efforts to do something which you can't 
already do well. The same goes for chess. Grandmasters, while they are very 
young, have played a huge number of games and also analyze the games they lost 
to eliminate all weaknesses in their gameplay.\\

Before Garry Kasparov, the then reigning world champion, was defeated by Deep 
Blue \cite{campbell2002deep} in May of 1997 in a chess match under tournament 
conditions, the two had a match in February 1996, when Kasparov came from 0-1 
down to win the match 4-2. The reason Kasparov could make a comeback was that 
he learned from Deep Blue's weaknesses and took advantage of them, while the 
deep blue computer responded similarly the next time too. This goes on to tell 
us the difference between a chess computer with no learning capability and a 
human grandmaster who learns about the weaknesses of the opponents. However, 
eventually Deep Blue was destined to beat Kasparov next year, but that was not 
an achieved because of a learning component, but more efficient and deeper 
search along with human intervention for tweaking the computer \cite{cnn-news}. 

\citet{gherrity1993game} introduced the notion of a general learning system, 
known as the \textit{Search and Learning} system (SAL), that could learn to 
play any rectangular board based two-player board game with a fixed types of 
pieces. SAL, trained using temporal difference learning, learns from the 
outcome of each game it plays. It was tested on Tic-tac-toe, Connect-4 and 
Chess. It could successfully play Tic-tac-toe and Connect-4, but the level of 
play he could achieve with a depth of search as 2 in chess was very poor. 
However, this inspired works like Neurochess \cite{thrun1995learning} which used 
chess game databases to learn the evaluation function while playing using a 
standard search-based algorithm.

\subsection{Chess Reasoning as Pattern Recognition}
\label{section:chess-as-pr}
The master level human chess players are known to recognize specific 
arrangements of the board or a specific set of pieces on the board and hence 
ignore certain poor positions to utilize their time to explore more important 
lines of play in a greater depth. We believe that an expert pattern recognition 
system, such as a human, would be able to recognize these patterns learned 
through experience and hence make a more informed decision to play a certain 
move. In other words, we think that a pattern recognition system that has seen 
enough examples of board positions and the corresponding expert moves should be 
able to predict favorable moves. But at the same time, we expect it to remain 
a necessity to explore those moves upto a greater depth.

\section{Problem statement}
In our work, we will take inspiration from these studies seeking a more human 
way of playing the game of chess rather than computer chess 
programs that predominantly employ search to make a move, which is a part of 
what humans do before making a move. In fact, we try to solve a tougher 
problem than just making a computer play chess well. We want the computer, 
which has seen enough number of games, to figure out the rules of the game 
without being explicitly programmed to know the rules of the game. By this 
task, we also try to account for the generalization capabilities of a human 
brain by looking at only a few examples of even a very vast set. For this task, 
we make use of Convolutional neural networks which is a specific artificial 
neural network architecture inspired by biological vision characteristically 
suited for pattern recognition tasks (described in greater detail 
in~\ref{subsection:cnn-background}). We want to emphasize that the our major 
motivation is not to build a system that can beat other state of the art 
systems, but to present a proof of concept that chess computers that are 
inspired by human thinking and pattern recognition, as observed in the 
case of grandmasters, can also yield a performance comparable to chess 
programs that are predominantly based on search.

\section*{Organization of the thesis}
The thesis is organized in the classic style of first introducing the problem 
(chapter~\ref{chap:introduction}), then presenting the background and related 
work (chapter~\ref{chap:background}), followed by chapters~\ref{chap:dataset} 
and \ref{chap:implementation} where we describe our dataset representation and 
implementation details respectively while systematically moving towards our 
aim. Finally we present the results and analysis (chapter~\ref{chap:results}). 
In the chapter~\ref{chap:introduction}, we specifically mention the 
convolutional neural network architecture we use, the loss functions and 
the how we employ them to choose a move while playing a game of chess. In the 
results chapter, we describe the strengths and weaknesses of our system by 
considering specific case studies. Finally, we conclude with a discussion about 
the feasibility of such systems and how they can be a part of the future of 
computer chess.